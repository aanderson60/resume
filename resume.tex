% E. Dunham -- Resume
% Contents Copyright (C) 2014 - 2016, E. Dunham

% LaTeX code for rendering the resume is distributed under the MIT license.
% See LICENSE.txt. It means you can use the code for whatever you want,
% including your own resume, but I'm not liable if it catches your computer on
% fire.

% Template originally developed by E. Dunham
% https://github.com/edunham/resume/blob/master/resume.tex

\documentclass[11pt]{article}
\usepackage[letterpaper]{geometry} %- letter size paper
\usepackage{graphicx}
% Some settings that the stuff hidden in the preamble will reference
\def \whitespaciness { 0 } % pt. bigger numbers = emptier page. 0 is ok.
\def \typeoffill {\dotfill} % hfill makes blank, dotfill makes dotted lines.


% E. Dunham -- Resume
% Contents Copyright (C) 2014 - 2016, E. Dunham

% LaTeX code for rendering the resume is distributed under the MIT license.
% See LICENSE.txt. It means you can use the code for whatever you want,
% including your own resume, but I'm not liable if it catches your computer on
% fire.

% Template originally developed by E. Dunham
% https://github.com/edunham/resume/blob/master/resume.tex
\usepackage[normalem]{ulem} % for the underlines
\usepackage[compact]{titlesec} % Shrink default spacings
\usepackage{tabto} % For aligning skills section
\usepackage{multicol} % for multicols command
\usepackage{ragged2e} % for /justify
\textwidth=7in
\textheight=10.5in
\topmargin -1in % Reclaim the default whitespace from top of page
\oddsidemargin -.25in % Reclaim whitespace on left, make it look centered
\pagenumbering{gobble} % Don't number pages
\setlength{\parindent}{0pt} % Don't indent paragraphs

\newcommand{\heading}[1]{
%    \section*{\centering\uline{\hfill #1 \hfill }} % Center the headings
    \section*{\uline{#1 \hfill}} % Right-align the headings
}
\newcommand{\squish}{
    \setlength{\parskip}{\whitespaciness pt}
}
\newcommand{\when}[1]{ % naming this 'date' would conflict with builtins
    \typeoffill \texttt{ #1}
}
\newcommand{\experience}[3]{ % place, optional title, date
    \ifx&#2&
        \item[{#1}]
    \else
        \item[{#1}, \emph{#2}]
    \fi
    \when{#3}
}
\newcommand{\event}[4]{
    \bf{#1} \tabto{2in} \texttt{#2} \tabto{3in} \normalfont
    \ifx&#3& \else
       \emph{ {#3},}
    \fi
    ``{#4}''
}
\newcommand{\contact}[4]{
    \centerline{ \large \texttt{ #1 $\bullet$ #2 $\bullet$ #3 }}
    \centerline{ \emph{ #4  $\bullet$ R\'{e}sum\'{e} current as of \today}}
}
\newcommand{\skill}[2]{
    \textbf{#1} \tabto{2in} #2
}
% Write C++ all fancy-like
% http://www.parashift.com/c++-faq-lite/latex-macros.html
\newcommand{\CPP}{
    C\hspace{-.05em}\raisebox{.4ex}{\tiny\bf +}\hspace{-.10em}\raisebox{.4ex}{\tiny\bf +}
}

% alternative to \skill, for extended lists of skills
% columnsep can be used here to unbalance the columns, with a negative number
% increasing the size of the right column versus the left.  '0cm' or equivalent
% will keep them balanced
%
% params: columnsep, heading, individual skills
\newcommand{\skillz}[3]{
    \vspace{-0.5cm}
    \squish
    \setlength{\columnsep}{#1}
    \begin{multicols}{2}
    \squish
    \RaggedRight % force to the hard left of the column
    \small
    \textbf{#2}
    \columnbreak
    \squish
    \justify
    \small
    #3
    \end{multicols}
    \vspace{-0.2cm} % yes, really necessary to keep this self-contained
}

% based on https://tex.stackexchange.com/a/148803
% intended to increase readability of longer entries under 'experience'
% params: indent/margin, item separation, top separation
\newenvironment{hangingparlist}[3]
    {\begin{list}
        {}
        {\setlength{\itemindent}{-#1}%%
        \setlength{\leftmargin}{#1}%%
        \setlength{\itemsep}{#2}%%
        \setlength{\parsep}{#2}%%
        \setlength{\topsep}{#3}%%
        }
    \setlength{\parindent}{-#1}%%
    \item[]
    }
    {\end{list}}

% Projects/Contributions
% params: name, optional website, description
\newcommand{\project}[3]{
    \begin{description}
        \setlength{\parsep}{1em}
        \ifx&#2&
            \item[#1] --- #3
        \else
            \item[#1]\emph{#2} --- #3
        \fi
    \end{description}
}

\begin{document}

\centerline{{\Huge \bf Alex Anderson}}

\bigskip

\contact{alexd43anderson@gmail.com}
        {aanderson60.github.io}
        {linkedin.com/in/aanderson60/}
        {Dallas, TX}

\heading{Academics}%%%%%%%%%%%%%%%%%%%%%%%%%%%%%%%%%%%%%%%%%%%%%%%%%%%%%%%%%%%%

\begin{description}
\squish
\experience{Texas A\&M University}
            {College Station, TX}
            {08/2019 - 05/2023}

Bachelor of Science in Electrical Engineering, Minor in Computer Science (GPA: 3.89) \\
Relevant Coursework: Analog VLSI Design, Fiber and Integrated Optics, Electronic Circuits

\end{description}

\heading{Experience}%%%%%%%%%%%%%%%%%%%%%%%%%%%%%%%%%%%%%%%%%%%%%%%%%%%%%%%%%%

\begin{description}
\squish
\experience{Undergraduate Research Assistant}
            {Texas A\&M University}
            {08/2022 - Present}

\underline{\smash{Analog and Mixed Signal Center}}: S. Palermo

\textbullet \space Design, verification, and measurement of a radiation-hardened optical transceiver in 180nm CMOS.

\textbullet \space Undergraduate thesis under University Research Scholars (URS) program.


\end{description}

\begin{description}
\squish
\experience{Applications Engineering Intern}
            {Texas Instruments}
            {06/2022 - 08/2022}

\underline{\smash{High-Speed Signal Conditioning Group}}: Dallas, TX
            
\textbullet \space Created internally and externally published documentation over TI family of USB 2.0 redrivers.

\textbullet \space Provided support, review, and debugging for customer designs and layouts. 

\textbullet \space Obtained lab measurements, compliance reports, and eye diagrams for redrivers, retimers, muxes.
      
\end{description}

\begin{description}
\squish
\experience{Undergraduate Research Assistant}
            {Texas A\&M University}
            {02/2022 - 06/2022}

\underline{\smash{Information Science Group}}: K. Narayanan, \underline{\smash{Microbiology and Food Safety}}: S. Pillai

\textbullet \space Developed unique experiments and testing schemes using group testing theory.

\textbullet \space Performed designed pooling experiments in a laboratory setting.

\textbullet \space Extensive simulation design in Python using packages including Scipy, Numpy, Matplotlib, Seaborn.

\end{description}

\heading{Publications and Presentations}

Y.-L. Luo, C. Hong, \textbf{A. Anderson}, D. Dolt, and S. Palermo, “A Radiation-Hardened Optical Transceiver in 180nm CMOS Technology,” accepted in \textit{2023 Government Microcircuit Applications and Critical Technology Conference.} \\

J. Molina, \textbf{A. Anderson}, S. Dixon, K. Narayanan, and S. Pillai, "A Single Stage Pooling Scheme for Large-Scale Pathogen Detection," submitted to \textit{Journal of Applied and Environmental Microbiology.} \\

\textbf{A. Anderson} and S. Dixon. Group Testing for Food Safety. Poster presented at: \textit{Student Research Week, College Station, TX (2022).}
\\

\heading{Skills}%%%%%%%%%%%%%%%%%%%%%%%%%%%%%%%%%%%%%%%%%%%%%%%%%%%%%%%%%%%%%%

\skill{Software}
      {Cadence Virtuoso, OrCAD, Allegro, LabVIEW, Linux, \LaTeX}

\skill{Programming}
      {MATLAB, Python,\CPP, HTML/CSS/Javascript}

\skill{Lab Tools}
      {Oscilloscope, Multimeter, Function Generator, Soldering}


\bigskip



\heading{Honors and Awards}%%%%%%%%%%%%%%%%%%%%%%%%%%%%%%%%%%%%%%%%%%%%%%%%%%%%%%%%%%

\begin{description}
\squish
\experience{Texas A\&M Engineering Honors}
           {College of Engineering}
           {08/2019 - Present}

\experience{Dean's Honor Roll}
           {College of Engineering}
           {05/2020, 12/2020, 05/2022}

\experience{McFadden Scholarship}
           {Texas A\&M University}
           {08/2019}

\experience{Eagle Scout}
           {Boy Scouts of America}
           {12/2016}

\end{description}

\end{document}
